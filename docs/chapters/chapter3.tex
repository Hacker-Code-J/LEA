\chapter{Mode of Operations Validation System (MOVS)}

\section{Structure}
\textcolor{blue}{\bf LEA128(CBC)KAT.txt:}
\begin{lstlisting}
KEY = 00000000000000000000000000000000
IV = 00000000000000000000000000000000
PT = 80000000000000000000000000000000
CT = CE8DCF04DD60982B1D8F5035FD534DE2
\end{lstlisting}
\vspace{4pt}
\textcolor{blue}{\bf LEA128(CBC)MMT.txt:}
\begin{lstlisting}
KEY = 724F8279123B9307658A109101466A15
IV = FEF995090770B941B0BF2B120A859BB8
PT = 76718FF5B60510FB4A9AA28CF9A57A60
CT = D65A9EC458B768C4E9BE62C6FE04A51D

KEY = AE38ECC785CC238F263D14285216B406
IV = BB0F694719D4BF967A085D4FD98A37E3
PT = F3F7057F5670F3E8BB9D9AAA95F12F71EA30FAB7622F0A9F9EDC2821CA7D0968
CT = D504FFC845EF447C516799AD8877F967628F49F15A8B64FA48AD215232884A52
\end{lstlisting}
\vspace{4pt}
\textcolor{blue}{\bf LEA128(CBC)MCT.txt:}
\begin{lstlisting}
COUNT = 0
KEY = 724F8279123B9307658A109101466A15
IV = FEF995090770B941B0BF2B120A859BB8
PT = 76718FF5B60510FB4A9AA28CF9A57A60
CT = 92E58A0B613CA94D922F7E6A445A4EC7
\end{lstlisting}
\vspace{12pt}

\begin{lstlisting}[style=C]
typedef struct {
	u32 key[4];     // Fixed 128 bits for key
	u32 iv[4];      // Fixed 128 bits for iv
	u32* pt;        // Pointer for arbitrary length plaintext
	size_t ptLength; // Length of pt
	u32* ct;        // Pointer for arbitrary length ciphertext
	size_t ctLength; // Length of ct
} CryptoData; // 64-byte (16 + 16 + 8 + 8 + 8 + 8)
\end{lstlisting}

\newpage
\subsection{\texttt{freeCryptoData()}}
\begin{lstlisting}[style=C]
void freeCryptoData(CryptoData* cryptoData) {
	if (cryptoData != NULL) {
		// Free the dynamically allocated memory for pt and ct
		free(cryptoData->pt);
		free(cryptoData->ct);
		
		// Set the pointers to NULL to avoid dangling pointers
		cryptoData->pt = NULL;
		cryptoData->ct = NULL;
		
		// Reset the lengths to zero
		cryptoData->ptLength = 0;
		cryptoData->ctLength = 0;
	}
}
\end{lstlisting}

\subsection{\texttt{parseHexLine()}}
\begin{lstlisting}[style=C]
void parseHexLine(u32* arr, const char* line) {
	for (int i = 0; i < 4; i++) {
		u32 value;
		sscanf(line + i * 8, "%8x", &value);
		*(arr + i) = value;
	}
}
\end{lstlisting}
\subsection{\texttt{parseHexLineVariable()}}
\begin{lstlisting}[style=C]
void parseHexLineVariable(u32* arr, const char* line, size_t length) {
	for (size_t i = 0; i < length; i++) {
		u32 value;
		// Ensure not to read beyond the line's end
		if (sscanf(line + i * 8, "%8x", &value) != 1) {
			// Handle parsing error, such as setting a default value or logging an error
			arr[i] = 0; // Example: set to zero if parsing fails
		} else {
			arr[i] = value;
		}
	}
}
\end{lstlisting}

\newpage
\subsection{\texttt{determineLength()}}
\begin{lstlisting}[style=C]
size_t determineLength(const char* hexString) {
	// Calculate the length of the hexadecimal string
	size_t hexLength = strlen(hexString);
	
	// Convert hex length to the length of the u32 array
	size_t u32Length = hexLength / 8;
	
	// If the hex string length is not a multiple of 8, add an extra element
	if (hexLength % 8 != 0) {
		u32Length++;
	}
	
	return u32Length;
}
\end{lstlisting}

\newpage
\subsection{\texttt{readCryptoData()}}
\begin{lstlisting}[style=C]
int readCryptoData(FILE* fp, CryptoData* cryptoData) {
	 // Assuming each line will not exceed this length
	char line[INITIAL_BUF_SIZE];
	
	while (fgets(line, sizeof(line), fp) != NULL) {
		if (strncmp(line, "KEY =", 5) == 0) {
			// Skip "KEY = "
			parseHexLine(cryptoData->key, line + 6);
		} else if (strncmp(line, "IV =", 4) == 0) {
		 	// Skip "IV = "
			parseHexLine(cryptoData->iv, line + 5);
		} else if (strncmp(line, "PT =", 4) == 0) {
		 	// Calculate length
			cryptoData->ptLength = determineLength(line + 5);
			cryptoData->pt = (u32*)malloc(cryptoData->ptLength * sizeof(u32));
			if (cryptoData->pt == NULL) return -1;
			parseHexLineVariable(cryptoData->pt, line + 5, cryptoData->ptLength);
		} else if (strncmp(line, "CT =", 4) == 0) {
			// Calculate length
			cryptoData->ctLength = determineLength(line + 5);
			cryptoData->ct = (u32*)malloc(cryptoData->ctLength * sizeof(u32));
			if (cryptoData->ct == NULL) {
				// Free pt if ct allocation fails
				free(cryptoData->pt);
				return -1;
			}
			parseHexLineVariable(cryptoData->ct, line + 5, cryptoData->ctLength);
		}
		// Add more conditions here if there are more data types
	}
	
	return 0; // Return 0 on successful read
}	
\end{lstlisting}

\newpage
\subsection{\texttt{compareCryptoData()}}
\begin{lstlisting}[style=C]
bool compareCryptoData(const CryptoData* data1, const CryptoData* data2) {
	// Compare fixed-size arrays: key and iv
	for (int i = 0; i < 4; i++) {
		if (data1->key[i] != data2->key[i] || data1->iv[i] != data2->iv[i]) {
			return 0; // Not equal
		}
	}
	
	// Compare lengths and contents of pt
	if (data1->ptLength != data2->ptLength) {
		return 0; // Not equal
	}
	for (size_t i = 0; i < data1->ptLength; i++) {
		if (data1->pt[i] != data2->pt[i]) {
			return 0; // Not equal
		}
	}
	
	// Compare lengths and contents of ct
	if (data1->ctLength != data2->ctLength) {
		return 0; // Not equal
	}
	for (size_t i = 0; i < data1->ctLength; i++) {
		if (data1->ct[i] != data2->ct[i]) {
			return 0; // Not equal
		}
	}
	
	return 1; // All comparisons passed, data structures are equal
}
\end{lstlisting}

\newpage
\section{Known Answer Test (KAT)}
\subsection{Overview}
\begin{lstlisting}[style=C]
void MOVS_LEA128CBC_KAT_TEST() {
	const char* folderPath = "../LEA128(CBC)MOVS/";
	char txtFileName[50]; char reqFileName[50];
	char faxFileName[50]; char rspFileName[50];
	
	// Construct full paths for input and output files
	snprintf(txtFileName, sizeof(txtFileName), "%s%s", folderPath, "LEA128(CBC)KAT.txt");
	snprintf(reqFileName, sizeof(reqFileName), "%s%s", folderPath, "LEA128(CBC)KAT.req");
	snprintf(faxFileName, sizeof(faxFileName), "%s%s", folderPath, "LEA128(CBC)KAT.fax");
	snprintf(rspFileName, sizeof(rspFileName), "%s%s", folderPath, "LEA128(CBC)KAT.rsp");
	
	create_LEA128CBC_KAT_ReqFile(txtFileName, reqFileName);
	create_LEA128CBC_KAT_FaxFile(txtFileName, faxFileName);
	create_LEA128CBC_KAT_RspFile(reqFileName, rspFileName);
	
	printf("\nLEA128-CBC-KAT-TEST:\n");	
	FILE* file1 = fopen(faxFileName, "r");
	FILE* file2 = fopen(rspFileName, "r");
	if (!file1 || !file2) {
		perror("Error opening files"); return;
	}
	
	CryptoData data1, data2;
	memset(&data1, 0, sizeof(CryptoData));
	memset(&data2, 0, sizeof(CryptoData));
	int result = 1; // Default to pass
	int idx = 1;
	int totalTests = 275; // Assuming a total of 275 tests
	int passedTests = 0;
	while (idx <= totalTests) {
		if (readCryptoData(file1, &data1) == -1 || readCryptoData(file2, &data2) == -1) {
			result = 0; // Indicate failure if read fails
			break;
		}
		
		if (!compareCryptoData(&data1, &data2)) {
			result = 0; // Fail
			printf("\nFAIL\n");
			break;
		}
		
		// Free the dynamically allocated memory
		freeCryptoData(&data1);
		freeCryptoData(&data2);
		
		// Reset the structures for the next iteration
		memset(&data1, 0, sizeof(CryptoData));
		memset(&data2, 0, sizeof(CryptoData));
		
		passedTests++;
		printProgressBar(idx++, totalTests);
	}

	printf("\n\nTesting Summary:\n");
	printf("Passed: %d/%d\n", passedTests, totalTests);
	if (result) {
		printf("Perfect PASS !!!\n");
	} else {
		printf("Some tests FAILED.\n");
	}
	
	fclose(file1);
	fclose(file2);
}
\end{lstlisting}

\subsection{\texttt{create\_LEA128CBC\_KAT\_ReqFile()}}
\begin{lstlisting}[style=C]
void create_LEA128CBC_KAT_ReqFile(const char* inputFileName, const char* outputFileName) {
	FILE *infile, *reqFile;
	char* line;
	size_t bufsize = INITIAL_BUF_SIZE;
	// Flag to check if it's the first KEY line
	int isFirstKey = 1; 
	
	// Open the source text file for reading
	infile = fopen(inputFileName, "r");
	if (infile == NULL) {
		perror("Error opening input file");
		return;
	}
	
	// Open the .req file for writing
	reqFile = fopen(outputFileName, "w");
	if (reqFile == NULL) {
		perror("Error opening .req file");
		fclose(infile);
		return;
	}
	
	// Allocate initial buffer
	line = (char*)malloc(bufsize * sizeof(char));
	if (line == NULL) {
		perror("Unable to allocate memory");
		fclose(infile);
		fclose(reqFile);
		return;
	}
	
	// Read the source file line by line
	while (fgets(line, bufsize, infile) != NULL) {
		if (strncmp(line, "KEY", 3) == 0) {
			if (!isFirstKey) {
				// If not the first KEY, add a newline before writing the line
				fputc('\n', reqFile);
			}
			isFirstKey = 0;
			fputs(line, reqFile);
		} else if (strncmp(line, "IV", 2) == 0 || strncmp(line, "PT", 2) == 0) {
			fputs(line, reqFile);
		}
	}
	
	// Free the allocated line buffer and close files
	free(line);
	fclose(infile);
	fclose(reqFile);
	
	printf("LEA128(CBC)KAT.req file has been successfully created in 'LEA128(CBC)MOVS' folder.\n");
}
\end{lstlisting}


\subsection{\texttt{create\_LEA128CBC\_KAT\_FaxFile()}}
\subsection{\texttt{create\_LEA128CBC\_KAT\_RspFile()}}

\newpage
\section{Multi-block Message Test (MMT)}

\newpage
\section{Monte Carlo Test (MCT)}




